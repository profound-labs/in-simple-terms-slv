
Učenje Ajahna Chahja, ki je že nekaj let deležno pozornosti tudi med slovenskimi theravadskimi budisti, v sodobnem času pridobiva vse večji ugled v okviru te budistične tradicije. Njegove globoke misli so izražene s preprostimi besedami in tako lahko koristijo vsakemu bralcu, ne glede na to, ali je izkušen meditant ali niti ne pozna Buddhovega učenja. Upam, da bo knjižica v pomoč tudi slovenskim bralcem, ki iščejo pot iz tegob in muk svojega vsakdana.

Zahvala za to objavo v slovenskem jeziku gre mnogim, ki želijo ostati anonimni, zlasti trem prevajalkam, ki so se potrudile, da bi bilo besedilo čim bliže izvornemu pomenu. Zahvala tudi vsem, ki so pomagali pri organizaciji, pregledovanju in urejanju. Hvala tudi samostanu Aruna Ratanagiri in sponzorjem iz skupine Kataññuta iz Malezije, Singapurja in Avstralije.

\vfill

{\par\raggedleft
Hiriko Bhikkhu\\
Samostan Cittaviveka, Anglija\\
\& Društvo Theravadskih Budistov Bhavana\\
Oktober 2012
\par}
