
Častiti Ajahn Chah je bil mojster prikladnih in nenavadnih prispodob za razlago Dhamme. Včasih je abstraktno misel ponazoril z živo in preprosto podobo; drugič je znal izvabiti iz podobe različne pomene na takšen način, da je bilo mogoče slutiti številne plasti pomena, o katerih je bilo treba dolgo razmišljati. Z drugimi besedami, nekatere od njegovih prispodob so bile že odgovor, medtem ko so druge porajale vprašanja.

Po njegovi smrti je bilo iz njegovih Dhamma-govorov zbranih kar nekaj zbirk prispodob. Prevod v knjigi, ki jo držite v rokah, temelji predvsem na zbirki, ki jo je v prvih letih tega desetletja uredil eden njegovih tajskih učencev, Ajahn Jandee. Pravim 'predvsem', ker sem sam dodal nekatere spremembe.

Tri prispodobe iz izvorne zbirke sem nadomestil s tremi drugimi, ki so bile del njegovega govora 'Razočarani nad tem, kar imate radi' (\emph{Byya khawng thii chawb}): 'Voda v steklenici, voda iz izvira', 'Ograja' in 'V obliki kroga'.  V dveh od teh primerov sta bili izvorni prispodobi zgolj podvajanje drugih prispodob iz zbirke. V tretjem primeru pa je bila izvorna prispodoba bolj historičnega kot praktičnega pomena.

Ena od izvornih prispodob – 'Voda kaplja, voda teče' – vsebuje nekaj dodatnih stavkov iz Dhamma-govora, v katerem je bila uporabljena.

Nekateri od naslovov prispodob so spremenjeni, da bi delovali učinkoviteje v prevodu.

Zaradi bolj povezane celote in toka pripovedi sem spremenil tudi vrstni red prispodob.

Ajahn Jandee je svojo zbirko zapisal po posnetkih Ajahn Chahjevih govorov, ne da bi bistveno posegal vanje, jaz pa sem poskusil slediti njegovemu pristopu tako, da sem izdelal kar se da natančen in neokrnjen prevod. Neizbrušenost nekaterih prispodob je ravno tisto, kar razkriva nepričakovane plasti pomena in zaradi česar so tak izziv – upam, da bo tudi ta prevod uspel pričarati nekaj te nedokončanosti, ki spodbuja k razmisleku.

Več ljudi je pregledalo izvorni rokopis in predlagalo koristne izboljšave. Zlasti bi se rad zahvalil Ajahnu Pasannu, Ginger Vathanasombat in Michaelu Zollu.

Naj vsi, ki berejo ta prevod, doumejo Ajahn Chahjev prvotni namen razložiti Dhammo s tako preprostimi in nazornimi besedami.

\vfill

{\par\raggedleft
Thanissaro Bhikkhu\\
SAMOSTAN METTA, ZDA\\
Oktober 2007
\par}
